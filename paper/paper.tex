\documentclass[10pt]{article}
\usepackage[margin=1in]{geometry} 
\usepackage{amsmath,amsthm,amssymb,amsfonts}
\usepackage{enumerate}
\usepackage{changepage}
\usepackage{graphicx}
\usepackage{tikz}
\usepackage{enumitem}
\usetikzlibrary{shapes,arrows,positioning}

\newcommand{\N}{\mathbb{N}}
\newcommand{\Z}{\mathbb{Z}}
\setlength\parindent{24pt}

\newif\ifquoteopen
\catcode`\"=\active % lets you define `"` as a macro
\DeclareRobustCommand*{"}{%
  \ifquoteopen
      \quoteopenfalse ''%
  \else
      \quoteopentrue ``%
  \fi
}

\newenvironment{problem}[2][]{\begin{trivlist}
  \item[\hskip \labelsep {\bfseries #1}\hskip \labelsep {\bfseries #2}]}{\end{trivlist}}

\begin{document}

\title{Senior Thesis Notes}
\maketitle

\begin{problem}{(Abstract)}
  As one augments their understanding of Haskell, he or she will come across the concept of Arrows.
  In this paper, we try to give a more simple definition of Arrows through the heavy use of working examples and their explanations. 
  We do this by first giving a good way of imagining the concept of Arrows through the concept of Monads. 
  Given this knowledge, we will demonstrate their relationship through a Kleisli Arrow and then give simple examples to show Arrows' efficiency. 
  We will finish this paper with a more sophisticated example that shows how useful arrows can be in a production environment.
\end{problem}

\begin{problem}{(Understanding Arrows)}
  Before we start with any example of arrows, we first need to explain what an Arrow is. 
  The best way to imagine this concept is by starting with a Monad.
  By definition, a Monad is a computation. 

\end{problem}

\begin{problem}{(Why is it that in the definition of arr, b and c are defined as different types when in fact they have to be the same in the (>>>)?)}
  This is because an arrow doesnt represent a particular function. 
  Instead, our arrow represents a set of functions with an abstract type of $b$ and $c$. 
  Each time you use the function $arr$, you actually create a new function that is in this set.
  So for example, if we have a function $A : Int -> String$ that say maps an integer to a string. 
  Then, we have a function $B : String -> Int$ that in the same set maps a string to an integer.
  These two functions fall into the same set. 
\end{problem}

\end{document}
